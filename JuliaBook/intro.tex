\chapter*{Introduction}
\addcontentsline{toc}{chapter}{Introduction}

\lettrine[nindent=0.35em,lhang=0.40,loversize=0.3]{T}{hese are the class notes} for the course ``Computational Physics'', lectured to the students of \textit{Bacharelado em Física de Materiais}, \textit{Física Médica}, and \textit{Licenciatura em Física} from the \textit{Instituto de Física}, \textit{Universidade Federal de Uberlândia} (UFU). Since this is an optional class intended for students with diverse knowledge and interest in Computational Physics and computer languages, I choose to discuss selected topics superficially with practical examples that may interest the students. Advanced topics are proposed as \textit{home projects} to the students accordingly to their interest. 

The ideal environment to develop the proposed lessons is the Linux operational system, preferably Debian-based distributions\footnote{Debian: \url{www.debian.org}, Ubuntu: \url{www.ubuntu.com}.}. However, for the lessons and examples of this notes we use the recently developed \texttt{Julia}\footnote{Julia: \url{www.julialang.com}.} language, which is open-source and available to any OS (GNU/Linux, Apple OS X, and MS Windows). This language has a syntax similar to MATLAB, but with many improvements. Julia was explicitly developed for numerical calculus focusing in vector and matrix operations, linear algebra, and distributed parallel execution. Moreover, a large community of developers bring extra functionalities to Julia via packages, \textit{e.g.:} the \texttt{PyPlot} package is plotting environment based on \texttt{matplotlib}, and the \texttt{ODE} package provides efficient implementation of adaptive Runge-Kutta algorithms for ordinary differential equations. Complementing Julia, we may discuss other efficient plotting tools like \texttt{gnuplot}\footnote{gnuplot: \url{www.gnuplot.info}.} and \texttt{Asymptote}\footnote{Asymptote: \url{asymptote.sourceforge.net}.}.

\textbf{Home projects.} When it comes to Computational Physics, each problem has a certain degree of difficulty, computational cost (processing time, memory, ...) and challenges to overcome. Therefore, the \textit{home projects} can be taken by group of students accordingly to the difficulty of the project. It is desirable for each group to choose a second programming language (\texttt{C/C++}, \texttt{Python}, ...) for the development, complementing their studies. The projects shall be presented as a short report describing the problem, computational approach, code developed, results and conclusions. Preferably, the text should be written in \LaTeX\xspace to be easily attached to this notes.

\textbf{Text-books.} Complementing these notes, complete discussions on the proposed topics can be found on the books available at UFU's library\cite{CScherer2010Metodos, NFranco2006Calculo, Arenales2008Calculo, JCButcher2008NumericalODE, thijssen2007computational, pang2006introduction}, and other references presented throughout the notes.

\vfill
Please check for an updated version of these notes at \url{www.infis.ufu.br/gerson}.