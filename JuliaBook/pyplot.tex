\section{PyPlot}
\label{sec:PyPlot}

\subsection{Installing PyPlot}

To install \texttt{PyPlot}, first you need Python and Matplotlib. If you use Ubuntu Linux, run

\begin{minted}{bash}
 $ sudo apt-get update
 $ sudo apt-get install python python-matplotlib
\end{minted}

Now you can install PyPlot. Start Julia and run: \mintinline[escapeinside=||]{julia}{|\julia| Pkg.add("PyPlot")}.

\subsection{Using PyPlot}

To use \texttt{PyPlot}, first you have to initialize it by calling: \mintinline[escapeinside=||]{julia}{|\julia| using PyPlot}. 

\begin{example}{Using PyPlot}
\begin{minted}{julia}
using PyPlot
x = linspace(-3pi,3pi,200);
plot(x,  sin(x)./x, color="red", linewidth=2.0, linestyle="-");
plot(x, -sin(x)./x, color="blue", linewidth=2.0, linestyle="--");
xlabel("x");
ylabel("f(x)");
\end{minted}
\end{example}

Since the \texttt{PyPlot} package is an interface to Python's Matplotlib, one may use its extensive documentation\footnote{Matplotlib's documentation: \url{http://matplotlib.org/}}.


\subsection{Most useful features and examples}

\subsubsection{Calling non-ported functions from Matplotlib}

PyPlot's documentation says that only the documented Python's \texttt{matplotlib.pyplot} API is exported to Julia. Nonetheless, other functions from Python's \texttt{PyPlot} can be accessed as

\begin{center}
 \texttt{matplotlib.pyplot.foo(...)} $\longrightarrow$ \texttt{plt[:foo](...)},
\end{center} 
where on the left we have Python's notation, and on the right Julia's.

For instance, you can generate an histogram using:

\begin{example}{Using non-ported functions from \texttt{Matplotlib}}
\begin{minted}{julia}
using PyPlot
x = randn(100000);
plt[:hist](x);
xlabel("Random number");
ylabel("Number of occurrences");
\end{minted}
\end{example}

\subsubsection{Latex}

It is very easy to use Latex with Julia's PyPlot package. Implicitly, PyPlot uses the LaTexStrings package\footnote{LaTeXStrings: \url{https://github.com/stevengj/LaTeXStrings.jl}}. Therefore on can use Latex commands on a string simply constructing it prepending with a \texttt{L}. For instance:

\begin{example}{Using Latex}
\begin{minted}[escapeinside=||,breaklines]{julia}
 using PyPlot
 
 x = linspace(0, 2pi, 100);
 f = sin(x);
 
 plot(x, f);
 xlabel(L"$\theta$ [rads]");
 ylabel(L"$\sin\theta$ [a.u.]");
\end{minted}
\end{example}

This Latex notation can be used in all text elements of PyPlot.

\subsubsection{Subplot}

The \texttt{subplot} command allows you to break the plot window into a grid. Say that the desired grid has $N$ lines and $M$ columns, the command \mintinline{julia}{subplot(N, M, j)} specifies that you want to plot the next figure into the $j$-th element of the grid, with the elements sorted in a ``Z'' shape. Try the next example.

\begin{example}{Using subplot}
\begin{minted}[escapeinside=||,breaklines]{julia}
 using PyPlot
 
 x = linspace(-5pi, 5pi, 1000);

 clf();
 
 subplot(2,2,1);
 plot(x, sin(x));
 xlabel(L"$x$");
 ylabel(L"$\sin(x)$");
 
 subplot(2,2,2);
 plot(x, cos(x));
 xlabel(L"$x$");
 ylabel(L"$\cos(x)$");

 subplot(2,2,3);
 plot(x, exp(-x.^2));
 xlabel(L"$x$");
 ylabel(L"Gaussian");
 
 subplot(2,2,4);
 plot(x, sin(x)./x); 
 xlabel(L"$x$");
 ylabel(L"$sinc(x)$");
 
 tight_layout(); # adjust the subplots to better fit the figure

\end{minted}
\end{example}

In the example above we are using Latex notation for the labels. The final command \textbf{tight\_layout(...)} allows you to adjust the subplots spacings to better adjust them within the figure. This avoids overlapping of the labels and plots.

Even more control can be achieved with the command \textbf{subplot2grid(...)}, which we use in Fig.~\ref{fig:FourierSeries}, for instance. Here the shape is passed as a tuple $(N,M)$ to the first parameter, the second parameter takes the location of the desired plot an ordered pair $(i,j)$, where $0 \leq i \leq (N-1)$ and $0 \leq j \leq (M-1)$. The top-left corner grid element is $(i,j) = (0,0)$, and the bottom-right is $(i,j) = (N-1, M-1)$. The \texttt{subplot2grid} becomes more interesting as the next parameters allow you to set a \texttt{rowspan} and a \texttt{colspan} to span over cells. Check again Example \ref{ex:FourierSeries}.


\subsubsection{Labels}

\red{Relevant commands: text}

\subsubsection{Legends}

\red{Relevant commands: legend}

\subsubsection{Other plot elements}

\red{Relevant commands: annotate, arrow, axes, axis, axhline, axvline, contour, }

\subsubsection{Saving into files (PNG, PDF, SVG, EPS, ...)}

Figures are shown by default on screen as PNG. However, you can save it to files with many different formats using the \texttt{savefig(...)} command. Check the example:

\begin{example}{Using subplot}
\begin{minted}[escapeinside=||,breaklines]{julia}
 using PyPlot
 
 x = linspace(-5pi, 5pi, 1000);
 f = sin(x);
 plot(x, f);
 xlabel(L"$\theta$ [rads]");
 ylabel(L"$f(\theta) = \sin\theta$");
 
 savefig("myplot.png");
 savefig("myplot.pdf");
 savefig("myplot.svg");
 savefig("myplot.eps");
\end{minted}
\end{example}

Please check the full documentation of the \texttt{matplotlib} for more details. You can choose the paper size, dpi, etc.

\subsubsection{Animations}

Here's an example of using Julia and \texttt{matplotlib} to create animations\footnote{Adapted from Andee Kaplan's presentation at \url{http://heike.github.io/stat590f/gadfly/andee-graphics/}}. The \texttt{PyCall} package is used to import the \texttt{animation} library from Python, since it is not natively implemented in \texttt{PyPlot}.

\begin{example}{Animated plot}
\begin{minted}[escapeinside=||,breaklines]{julia}
using PyCall # package used to call Python functions
using PyPlot

# import the animation library
@pyimport matplotlib.animation as anim

# this function must plot a frame of the animation
# the parameter t will be iterated by the matplotlib
function animate(t)
  # in this example we willl plot a Gaussian moving around
  c = 3*sin(2pi*t);
  x = linspace(-10, 10, 200);
  y = exp(-(x-c).^2/2);

  # clear the figure before ploting the new frame
  clf(); 
  
  # plot the new frame
  plot(x,y)

  # for this basic usage, the returned value is not important
  return 0
end

# matplotlib's FuncAnimation require an explicit figure object
fig = figure();

tspan = linspace(0, 5, 100); # will be used to iterate the frames

# animation with a interval of 20 miliseconds between frames
video = anim.FuncAnimation(fig, animate, frames=tspan, interval=20)

# save the animation to a MP4 file using ffmpeg and the x264 codec
video[:save]("anim.mp4", extra_args=["-vcodec", "libx264"])
\end{minted}
\end{example}













